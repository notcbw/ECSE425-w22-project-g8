% compile: pdflatex report.tex
\documentclass[lettersize,journal]{IEEEtran}
\usepackage{amsmath,amsfonts}
\usepackage{algorithmic}
\usepackage{algorithm}
\usepackage{array}
\usepackage[caption=false,font=normalsize,labelfont=sf,textfont=sf]{subfig}
\usepackage{textcomp}
\usepackage{stfloats}
\usepackage{url}
\usepackage{verbatim}
\usepackage{graphicx}
\usepackage{cite}
\hyphenation{op-tical net-works semi-conduc-tor IEEE-Xplore}
% updated with editorial comments 8/9/2021

\begin{document}

\title{Final Project Report}

\author{Krishna Aroomoogon, Bowen Cui, Kaan G\"{u}re, Amine Mallek, James Willems}

% The paper headers, uncomment if you want to use it
%\markboth{Journal of \LaTeX\ Class Files,~Vol.~14, No.~8, August~2021}%
%{Shell \MakeLowercase{\textit{et al.}}: A Sample Article Using IEEEtran.cls for IEEE Journals}

\maketitle

%\begin{abstract}
%This document describes the most common article elements and how to use the IEEEtran class with \LaTeX \ to produce files %that are suitable for submission to the IEEE.  IEEEtran can produce conference, journal, and technical note %(correspondence) papers with a suitable choice of class options. 
%\end{abstract}

\section{Introduction}
\IEEEPARstart{T}{his} is the project report for the implementation of a standard five-stage pipelined 32-bit MIPS processor in VHDL.

\section{Methodology}
% write something in general?
\subsection{Instruction Fetch}

\subsection{Instruction Decode}

\subsection{Execute}

\subsection{Memory}

\subsection{Write-back}


\section{Results and Discussion}

\section{Optimization}

\section{Limitations}

\section{Conclusion}



\end{document}


